\documentclass[main.tex]{subfiles}

\begin{document}

\section{Appendix B - Maintenance Manual}
\label{maintance-manual}
\label{MaintenanceManual}

\subsection{Prerequisites}
\label{prerequisits-for-running-thalia}

Before running Thalia on a local setup, the following prerequisites must be fulfilled:

\begin{itemize}
\item
  The project must be run on a supported operating system, i.e. one of: MacOS, Ubuntu, AWS Linux 2, or Debian. While it also seems to run on Windows 10, there are no strict correctness guarantees as the test suite has been developed exclusively for UNIX flavoured operating systems.
\item
  Python \textgreater= 3.8 must be available.
\item
  The Python virtual environment module `venv` and the `pip` package installer must be installed.
\item
  One of either Chrome or Firefox must be available for accessing the website.
\item
  To run Selenium tests, you will need both Chrome and Firefox installed, with both `geckodriver` and `chromedriver` available in order to run tests in headless mode.
  You can find more information about this in the Selenium directory's `README.md`.
\end{itemize}

\subsection{Installation}
\label{installation-instructions}

In order to install Thalia, execute the following instructions in order of listing. The instructions assume a UNIX shell. If you're running Windows, please substitute instructions as needed.

\begin{enumerate}

\item
  \texttt{cd} into the project directory
\item
  Create a virtual environment (\urllink{https://docs.python.org/3/tutorial/venv.html}{https://docs.python.org/3/tutorial/venv.html}) and activate it: \newline
  
  \texttt{python3\ -m\ venv\ venv\ \&\&\ source\ ./venv/bin/activate}
\item Install the Python dependencies via pip
(\urllink{https://docs.python.org/3/installing/index.html}{https://docs.python.org/3/installing/index.html}):
\newline
  \texttt{pip\ install\ -r\ requirements.txt}
\end{enumerate}

\subsubsection{Registering a Database}

There are two ways to create a database of financial assets: use a pre-made database \texttt{.db} file or create your own database by using the Harvester.

\subsubsection*{Using an existing Database}
\label{existing-db-instructions}

Place the database into the `Finda` directory. Rename it to \texttt{asset.db}. You may now register it with Finda by running the \texttt{register\_findb.py} script.

\subsubsection*{Creating a Database with the Data Harvester}
\label{make-your-own-db-instructions}

Create your own database with the Data Harvester will take some time due to the large number of API calls involved. You will need an API key for the Nomics API. Once you have acquired the necessary key, place it in a file called `nomics` within the `DataHarvester/apis\_access` directory. Then, open a new terminal and follow these instructions:
\begin{enumerate}
  \item Run \texttt{python\ first\_run\_dh.py} to create, register and seed the financial data database.
\item
  Start the harvester with \texttt{python\ run\_harvester.py},
  which will run forever.
\item
  To see what assets have been updated, have a look at the
  `DataHarvester/persistent\_data` directory.
\item
  You can start and stop the Harvester at any time. When
  restarted, it will continue the updating session as long as any updates remain.
\end{enumerate}
Once all the assets have been updated at least once, you can proceed to the next section. This may take up to one day since we must abide by contractual API limits.

\subsection{Running the application in a Development Environment}
  To run the project in a development environment from a modern UNIX shell, you can use the \texttt{start.sh} script. After running it, you will be able to access the website from \texttt{localhost:5000}.

\subsubsection{Running the application in a Production
Environment}
\label{run-in-a-production-environemnt}

Procedures to run the application in production differ from the ones listed for a development environment. Follow these instructions to setup Thalia on a web server.

\begin{itemize}
\item
  Install the `gunicorn` HTTP server \cite{Gunicorn}. If you have installed the project's requirements from `requirements.txt`, this should already be available in your virtual environment.
\item
  Run the following command from a UNIX terminal 
  \newline
  \texttt{sudo\ gunicorn\ -\/-bind\ 0.0.0.0:80\ -\/-daemon\ main:server}
\end{itemize}
The website should now be accessible via the standard HTTP port 80 from your IP address.

\subsection{Test instructions}
\label{test-instructions}

The following instructions have to be executed to test the application:

\begin{enumerate}
 \item If you have not done so already, install the project requirements by using
 \newline
 \texttt{pip\ install\ -r\ requirements.txt}
 
 \item You will also need to install additional development and testing dependencies by using
 \newline
 \texttt{pip\ install\ ".{[}test,dev{]}"}
  
  \item Python and Selenium tests must be run separately. The former can be achieved by issuing
  \newline
  \texttt{python\ -m\ pytest\ -\/-ignore=./Tests/Selenium}
  
  \item To run Selenium tests, run the application in a development environment to ensure it is available at \texttt{localhost:5000}. You may then test it by using
  \newline
  \texttt{python\ Tests/Selenium/main\_test.py}
\end{enumerate}

\subsection{Source Code Files}
\label{source-code-files}

The following sections explain the responsibilities of every source code file. 

\subsection{Anda}
\label{anda}

\begin{itemize}
\item
  analyse\_data/analyse\_data.py -- The library that handles all backtesting calculations.
\end{itemize}

\paragraph{Tests}\label{tests}

\begin{itemize}

\item
  Tests/analyse\_data/test\_anda.py -- Tests for analyse\_data.py
\item
  Tests/analyse\_data/test\_data -- Test data used by test\_anda.py
\end{itemize}

\subsection{Data Harvester}
\label{data-harvester}

\begin{itemize}
\item
  DataHarvester/apis\_access/ -- Contains all our API keys for
  the Harvester's raw data retrieval.
\item
  DataHarvester/dhav\_core/api\_class.py -- Abstracts individual API service methods for data retrieval from other parts of code.
\item
  DataHarvester/dhav\_core/data\_harvester.py -- Main module for the package.
\item
  DataHarvester/dhav\_core/init\_finda\_db\_structure.py -- Seeds
  database with current assets and metadata.
\item
  DataHarvester/dhav\_core/init\_update\_list.py -- Script for resetting the accumulated persistent data.
\item
  DataHarvester/dhav\_core/initialization.py -- Initial setup required for other modules in package.
\item
  DataHarvester/dhav\_core/logger\_class.py -- Logging for Harvester.
\item
  DataHarvester/dhav\_core/run\_updates.py -- Script used to setup API keys and batch sizes for a Harvester update cycle.
\item
  DataHarvester/dhav\_core/tickers/ -- Directory contains the source files for assets in different assets classes. The files dictate our supported assets in production.
\item
  run\_harvester.py -- Script for running harvester on a timed interval.
\item
  first\_run\_dh.py -- Data Harvester Initializer, run once.
\end{itemize}

\paragraph{Tests}
\label{tests-1}

\begin{itemize}

\item
  Tests/Harvester/test\_api\_caller.py -- Tests for API handling.
\item
  Tests/Harvester/test\_interpolation.py -- Tests for financial data interpolation.
\item
  Tests/Harvester/interpolation\_result.csv -- Expected interpolation results.
\item
  Tests/Harvester/to\_interpolate.csv -- Data to be interpolated during tests.
\end{itemize}

\subsection{Finda}\label{finda}

\begin{itemize}

\item
  Finda/asset.db -- If only using the basic setup scripts, asset.db is the assumed main financial database.
\item
  Finda/dbSchema.sql -- SQL schema for the financial data database.
\item
  Finda/fd\_manager.py -- Database controller for the financial data database.
\item
  Finda/fd\_read.py -- Methods for reading from the financial data database.
\item
  Finda/fd\_remove.py -- Methods for deleting from financial data
  database.
\item
  Finda/fd\_write.py -- Methods for writing to financial database.
\item
  register\_findb.py -- Script used for ensuring the financial database has been registered, run once.
\end{itemize}

\paragraph{Tests}\label{tests-2}

\begin{itemize}

\item
  Tests/Finda/conftest.py -- Setup and tear down code for Finda tests.
\item
  Tests/Finda/finData.db -- Test database.
\item
  Tests/Finda/helpers.py -- Testing helper methods for Finda.
\item
  Tests/Finda/test\_fd\_manager.py -- Tests for fd\_manager.
\item
  Tests/Finda/test\_fd\_read.py -- Tests for fd\_read.
\item
  Tests/Finda/test\_fd\_remove.py -- Tests for fd\_remove.
\item
  Tests/Finda/test\_fd\_write.py -- Tests for fd\_write.
\end{itemize}

\subsection{Thalia}\label{thalia}

\subsubsection{Dashboard}\label{dashboard-specific}

\begin{itemize}

\item
  Thalia/dashboard/callbacks/allocations.py -- Controller for portfolio creation, saving and retrieval.
\item
  Thalia/dashboard/callbacks/assets.py -- Controller for the asset contribution tab.
\item
  Thalia/dashboard/callbacks/callbacks.py -- Collects all other
  callbacks for easy importing.
\item
  Thalia/dashboard/callbacks/dashboard.py -- Controller for running a backtest and all side effects of running a backtest.
\item
  Thalia/dashboard/callbacks/drawdowns.py -- Controller for the
  drawdowns tab.
\item
  Thalia/dashboard/callbacks/metrics.py -- Controller for the metrics tab.
\item
  Thalia/dashboard/callbacks/overfitting.py -- Controller for
  overfitting check.
\item
  Thalia/dashboard/callbacks/returns.py -- Controller for returns tab.
\item
  Thalia/dashboard/callbacks/summary.py -- Controller for the summary tab.
\item
  Thalia/dashboard/tab\_elements/ -- Directory contains UI generation code.
\item
  Thalia/dashboard/tab\_elements/elements.py -- Generic elements shared by multiple tabs.
\item
  Thalia/dashboard/tab\_elements/lazy\_portfolios -- A list of formatted pre-made portfolios.
\item
  Thalia/dashboard/user-data -- Directory contains the temporary user uploaded CSVs.
\item
  Thalia/dashboard/config.py -- Commonly used constants used in the dashboard codebase.
\item
  Thalia/dashboard/layout.py -- Declaration of the main UI elements.
\item
  Thalia/dashboard/portfolio\_manager.py -- Methods for storing and retrieving portfolios from the database.
\item
  Thalia/dashboard/strategy.py -- Adapter for passing user input to Anda.
\item
  Thalia/dashboard/tabs.py -- Tab element configuration and declarations.
\item
  Thalia/dashboard/user\_csv.py -- Back-end code for handling temporary user uploaded CSVs.
\item
  Thalia/dashboard/util.py -- Methods for accessing Finda in the
  dashboard.
\end{itemize}

\paragraph{Tests}\label{tests-3}

\begin{itemize}

\item
  Tests/Thalia/callbacks/test\_allocations.py -- Tests for allocation tab.
\item
  Tests/Thalia/callbacks/test\_dashboard.py -- Tests for
  elements across dashboard.
\item
  Tests/Thalia/callbacks/test\_summary.py -- Tests for summary tab.
\item
  Tests/Thalia/test\_user\_csv.py -- Tests for user uploaded CSV
  handling.
\item
  Tests/Thalia/test\_portfolio.py -- Tests for portfolio storage.
\end{itemize}


\subsubsection{Flask}\label{flask}

\begin{itemize}

\item
  Thalia/models/portfolio.py -- The Portfolio ORM module. Handles user saved portfolio interactions with the database.
\item
  Thalia/models/user.py -- The User account ORM module. Handles user account interactions with the database.
\item
  Thalia/static/ -- Contains static assets, images, scripts, and stylesheets used in the Flask parts of the website.
\item
  Thalia/templates/ -- Contains Jinja2 templates for different parts of the Flask website. Includes the base, homepage, authentication page, info pages, and user portfolio gallery templates.
\item
  Thalia/\_\_init\_\_.py -- Module responsible for the app factory of the Flask app. Passes complete Flask apps to web workers such as gunicorn. Includes Flask app setup, configuration, dash app registrations and protection, database registrations, extension setup.
\item
  Thalia/extensions.py -- Module for initialising different Flask
  extensions.
\item
  Thalia/findb\_conn.py -- Module for accessing Finda. Separated from other modules to avoid circular imports.
\item
  Thalia/forms.py -- Declarations of different HTML form controllers.
\item
  Thalia/views.py -- Main UI logic for Flask code. Handles routing and controlling user interactions with the Flask code.
\item
  app.dbexport -- The user data database.
\item
  config.py -- Flask app configuration declarations. Includes
  development defaults in case user has not defined the necessary
  environmental variables.
\item
  feedback.csv -- User submitted feedback from the website.
\item
  main.py -- Used as the `main` method for the project. It provides the results of the app factory to be passed to a server.
\end{itemize}

\paragraph{Tests}\label{tests-4}

\begin{itemize}

\item
  Tests/Thalia/conftest.py -- Setup and teardown code for Flask tests.
\item
  Tests/Thalia/test\_factory.py -- Test app factory pattern.
\item
  Tests/Thalia/test\_user.py -- Test user account handling.
\item
  Tests/Thalia/test\_views.py -- Integration tests for views.py.
\end{itemize}

\subsubsection{Selenium Tests}\label{selenium-tests}

\begin{itemize}

\item
  Tests/Selenium/contact\_test.py -- Test contact form.
\item
  Tests/Selenium/dashboard\_test.py -- Test dashboard.
\item
  Tests/Selenium/login\_test.py -- Test login.
\item
  Tests/Selenium/main\_test.py -- Convenience script to run all Selenium tests.
\item
  Tests/Selenium/navbar\_test.py -- Test navigation bar.
\item
  Tests/Selenium/README.md -- Instructions and info on Selenium tests.
\item
  Tests/Selenium/register\_test.py -- Registration tests.
\item
  Tests/Selenium/social\_test.py -- Tests for social media links.
\item
  Tests/Selenium/util.py -- Utilities used by other Selenium tests.
\item
  Tests/Selenium/webdriver\_test.py -- Test for checking whether Selenium configuration is working
\end{itemize}

\subsection{Miscellaneous}\label{misc-files}

\begin{itemize}

\item
  .circleci/config.yml -- Defines our continuous integration jobs for CircleCI.
\item
  .flake8 -- Style guide rules for linting Python code.
\item
  pyproject.toml -- Style rules for automatic code formatter.
\item
  pytest.ini -- Default rules for pytest to run our tests.
\item
  README.md -- Short instructions on installation, tests and other notes.
\item
  requirements.txt -- Our project's python requirements with versions locked and tested for compatibility.
\item
  setup.py -- Python setup file with more loose requirements. Includes requirements necessary for testing and contributing.
\item
  start.sh -- Commonly used script for running the project in a development environment.
\item
  .pre-commit-config.yaml -- Pre-commit hook.
\end{itemize}

\end{document}
