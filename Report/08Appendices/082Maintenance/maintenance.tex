\documentclass[main.tex]{subfiles}
\ProvidesPackage{preamble}

\usepackage[nottoc]{tocbibind}
\usepackage[english]{babel}
\usepackage[utf8]{inputenc}
\usepackage[table]{xcolor}
\usepackage[nohead, nomarginpar, margin=1in, foot=.25in]{geometry}
\usepackage{tabularx}
\usepackage{graphicx}
\usepackage{float}
\usepackage[english]{babel}
\usepackage{paralist}
\usepackage{datetime}
\usepackage{afterpage}

\usepackage{listings}
\usepackage{xcolor}

\usepackage{dirtytalk}


\begin{document}

\subsection{Appendix B - Maintenance Manual}
\label{MaintenanceManual}
\section{Maintance manual}\label{maintance-manual}

\subsection{Prerequisits for running
Thalia}\label{prerequisits-for-running-thalia}

\begin{itemize}

\item
  Project has been development to be run on a unix operating system. It
  has been tested to work without problems on MacOS, Ubuntu, AWS Linux
  2, Debian Jesse based docker images, and although technically works on
  Windows 10 some issues have been found in running test, so we
  recommend running
\item
  Python \textgreater=3.8
\item
  Python virtual environement module and pip modules
\item
  For testing the website: latest version of Chrome or Firefox to ensure
  all features work on the website
\item
  For Selenium tests you will need both Chrome and Firefox installed,
  with geckodriver and chromedriver to run the tests in headless mode.
  You can find more information about this in the Selenium directories
  README.md
\end{itemize}

\subsection{Installation instructions}\label{installation-instructions}

\begin{enumerate}
\def\labelenumi{\arabic{enumi}.}

\item
  \texttt{cd} into the project directory
\item
  create a virtual environment and activate it
  (https://docs.python.org/3/tutorial/venv.html)

  \begin{itemize}

  \item
    your operating system may not have virtual environment installed by
    default and you may need to install it seperately yourself
  \item
    assuming everything works the usual argument for creation and
    activation would be:
    \texttt{python3\ -m\ ven\ venv\ \&\&\ source\ ./venv/bin/activate}
    for a modern unix shell
  \end{itemize}
\item
  install the python dependencies:
  \texttt{pip\ install\ -r\ requirements.txt}

  \begin{itemize}

  \item
    pip may also not be installed on your operating system by default,
    so you may need to install it seperately yourself
  \end{itemize}
\end{enumerate}

\subsubsection{Existing DB
instructions}\label{existing-db-instructions}

\begin{enumerate}
\def\labelenumi{\arabic{enumi}.}

\item
  If you're using an premade financial database and are not using
  Harvester to create it yourself, then please ensure the that the file
  is inside the Finda directory.
\item
  Register the financial database with Finda, if the files name is
  \texttt{asset.db} this can be done automatically by running the
  \texttt{register\_findb.py} script.
\item
  To run the project in a development environment on a modern unix
  shell, you can use the \texttt{start.sh} script, you should then be
  able to see the website on your browser at the location:
  \texttt{localhost:5000}.
\end{enumerate}

\subsubsection{Make your own DB
instructions}\label{make-your-own-db-instructions}

\begin{enumerate}
\def\labelenumi{\arabic{enumi}.}

\item
  If you want to create your own database with the Data Harvester
  module, you save a bit of time to gather the data and possibly open a
  new shell instance so you can leave it running. You will need an API
  key from yahoo finance and nomics for Harvester to gather all the
  necessary assets. Both are free
  (https://p.nomics.com/cryptocurrency-bitcoin-api)
\item
  First run \texttt{python\ first\_run\_dh.py} to create, register and
  seed the database.
\item
  Then you can start harvester with \texttt{python\ run\_harvester.py},
  which will run forever, but you can check up on how it's doing by
  reading the log-file with somethign like \texttt{tail\ -F\ dh.log}.
  Once all the assets have been updates at least once (this may take a
  day as we don't want to bombard our financial data providers with too
  many requests too fast), you can proceed to use Thalia.
\item
  You can run the app now with the 3rd step from the ``Existing DB
  instructions''
\end{enumerate}

Notes:

\begin{itemize}

\item
  To see what assets have been updated and for what dates go in the
  persistent\_data folder inside the DataHarvester.
\item
  You can start and stop the Harvester at any time you want. When
  restarted it will continue the updating session if any updates remain.
\item
  To modify the time in between update sessions go into the
  run\_harvester.py file and modify the sleep time.
\end{itemize}

\subsubsection{Run in a production
environemnt}\label{run-in-a-production-environemnt}

\begin{itemize}

\item
  We recommend using \texttt{gunicorn} to run the website in production.
\item
  Please ensure that you set environmental variables for production
  appropriate settings, you can find the env vars we use inside
  config.py.
\item
  An easy way to set it up is running just by running the following
  command
  \texttt{sudo\ gunicorn\ -\/-bind\ 0.0.0.0:80\ -\/-daemon\ main:server}.

  \begin{itemize}

  \item
    you may need to provide the path to the gunicorn binary as an
    absolute path.
  \end{itemize}
\item
  The website should now be accessible on your IP address without
  specifying the port number and ready for production.
\end{itemize}

\subsection{Test instructions}\label{test-instructions}

\begin{itemize}
\item
  The project uses pytest (https://docs.pytest.org/) to run most test,
  with the exception of Selenium tests, which we have a seperate script
  for running.

  \begin{enumerate}
  \def\labelenumi{\arabic{enumi}.}
  \item
    To run the tests you should ensure you have the proper requirements
    first installed from the requirements.txt file
    \texttt{pip\ install\ -r\ requirements.txt}
  \item
    Then install the development and testing requirements with
    \texttt{pip\ install\ ".{[}test,dev{]}"}
  \item
    The run the python tests with
    \texttt{python\ -m\ pytest\ -\/-ignore=./Tests/Selenium}
  \item
    After you can run the Selenium tests by first starting the website
    normally and ensuring it's available on http://127.0.0.1:5000, then
    running \texttt{python\ Tests/Selenium/main\_test.py}
  \end{enumerate}
\end{itemize}

\subsection{Source code files}\label{source-code-files}

\subsubsection{Anda}\label{anda}

\begin{itemize}

\item
  analyse\_data/analyse\_data.py -- The financial calculation library
  that handles all backtesting calculations
\end{itemize}

\paragraph{Tests}\label{tests}

\begin{itemize}

\item
  Tests/analyse\_data/test\_anda.py -- Tests for analyse\_data.py
\item
  Tests/analyse\_data/test\_data -- Test data used by test\_anda.py
\end{itemize}

\subsubsection{Data Harvester}\label{data-harvester}

\begin{itemize}

\item
  DataHarvester/apis\_access/ -- Contains all our API keys for
  harvesters raw data retrieval
\item
  DataHarvester/dhav\_core/api\_class.py -- Abstracts individual API
  services methods for data retrieval from other parts of code
\item
  DataHarvester/dhav\_core/data\_harvester.py -- Main module for the
  package, which manages jobs and the other modules and stores data
\item
  DataHarvester/dhav\_core/init\_finda\_db\_structure.py -- Seeds
  database with current assets and metadata Harvester knows
\item
  DataHarvester/dhav\_core/init\_update\_list.py -- Script for resetting
  the persistent data the harvester has accumulated while running
\item
  DataHarvester/dhav\_core/initialization.py -- Does initial setup for
  other modules in package to function
\item
  DataHarvester/dhav\_core/logger\_class.py -- Basic logging class for
  Harvester
\item
  DataHarvester/dhav\_core/run\_updates.py -- Basic script used to setup
  api keys and batch sizes for a harvester update cycle
\item
  DataHarvester/dhav\_core/tickers/ -- Directory contains the source
  files for assets in different assets classes. The files dictate our
  supported assets in production
\item
  run\_harvester.py -- script for running harvester on a timer, should
  be made into a daemon in production
\item
  first\_run\_dh.py -- initialising Harvester for the project to be run
  only once
\end{itemize}

\paragraph{Tests}\label{tests-1}

\begin{itemize}

\item
  Tests/Harvester/test\_api\_caller.py -- tests for api handling
\item
  Tests/Harvester/test\_interpolation.py -- tests financial data
  interpolation
\item
  Tests/Harvester/interpolation\_result.csv -- expected interpolation
  results
\item
  Tests/Harvester/to\_interpolate.csv -- data to be interpolated for
  tests
\end{itemize}

\subsubsection{Finda}\label{finda}

\begin{itemize}

\item
  Finda/asset.db -- if only using the basic setup scripts asset.db is
  the assumed main financial database
\item
  Finda/dbSchema.sql -- SQL schema for the financial database
\item
  Finda/fd\_manager.py -- Controler for the financial data database
\item
  Finda/fd\_read.py -- Methods for reading from financial data database
\item
  Finda/fd\_remove.py -- Methods for deleting from financial data
  database
\item
  Finda/fd\_write.py -- Methods for writing to financial database
\item
  register\_findb.py -- script used for ensuring the financial database
  has been registered, only ever needs to be run once
\end{itemize}

\paragraph{Tests}\label{tests-2}

\begin{itemize}

\item
  Tests/Finda/conftest.py -- setup and tear down code for Finda tests
\item
  Tests/Finda/finData.db -- test database
\item
  Tests/Finda/helpers.py -- testing helper methods for Finda
\item
  Tests/Finda/test\_fd\_manager.py -- tests for fd\_manager
\item
  Tests/Finda/test\_fd\_read.py -- tests for fd\_read
\item
  Tests/Finda/test\_fd\_remove.py -- tests for fd\_remove
\item
  Tests/Finda/test\_fd\_write.py -- tests for fd\_write
\end{itemize}

\subsubsection{Thalia}\label{thalia}

\paragraph{Dashboard specific}\label{dashboard-specific}

\begin{itemize}

\item
  Thalia/dashboard/callbacks/allocations.py -- Controller for portfolio
  creation, saving and loading
\item
  Thalia/dashboard/callbacks/assets.py -- Controller for the asset
  contribution tab
\item
  Thalia/dashboard/callbacks/callbacks.py -- Collects all other
  callbacks for easy importing
\item
  Thalia/dashboard/callbacks/dashboard.py -- Controller for running a
  backtest and side effects of running a backtest
\item
  Thalia/dashboard/callbacks/drawdowns.py -- Controller for the
  drawdowns tab
\item
  Thalia/dashboard/callbacks/metrics.py -- Controller for the metrics
  tab
\item
  Thalia/dashboard/callbacks/overfitting.py -- Controller for
  overfitting checking
\item
  Thalia/dashboard/callbacks/returns.py -- Controller for returns tab
\item
  Thalia/dashboard/callbacks/summary.py -- Controller for the summary
  tab
\item
  Thalia/dashboard/tab\_elements/ -- Directory contains UI generation
  code, with the exception of the modules are seperated to handle an
  entire individual tabs contents, which mirrors its controller's name.
\item
  Thalia/dashboard/tab\_elements/elements.py -- Generic elements shared
  by multiple tabs
\item
  Thalia/dashboard/tab\_elements/lazy\_portfolios -- A list of formatted
  pre-made portfolios
\item
  Thalia/dashboard/user-data -- Directory contains the temporary user
  uploaded CSVs
\item
  Thalia/dashboard/config.py -- Commonly used constants used in the
  dashboard codebase
\item
  Thalia/dashboard/layout.py -- Declaration of the main UI elements
\item
  Thalia/dashboard/portfolio\_manager.py -- Methods for storing and
  retrieving portfolios from the database
\item
  Thalia/dashboard/strategy.py -- Adapter for passing user input to Anda
\item
  Thalia/dashboard/tabs.py -- Tab element configuration and declarations
\item
  Thalia/dashboard/user\_csv.py -- Backend code for handling temporary
  user uploaded CSVs
\item
  Thalia/dashboard/util.py -- Methods for accessing Finda in the
  dashboard
\end{itemize}

\paragraph{Tests}\label{tests-3}

\begin{itemize}

\item
  Tests/Thalia/callbacks/test\_allocations.py -- tests for allocation
  tab
\item
  Tests/Thalia/callbacks/test\_dashboard.py -- general tests for
  elements across dashboard
\item
  Tests/Thalia/callbacks/test\_summary.py -- tests for summary tab
\item
  Tests/Thalia/test\_user\_csv.py -- tests for user uploaded csv
  handling
\item
  Tests/Thalia/test\_portfolio.py -- test for portfolio storage
\end{itemize}


\subsubsection{Flask}\label{flask}

\begin{itemize}

\item
  Thalia/models/portfolio.py -- The Portfolio ORM module, handles user
  saved portfolio interactions with the database
\item
  Thalia/models/user.py -- The User account ORM module, handles user
  account interactions with the database
\item
  Thalia/static/ -- Contains static assets images, scripts, stylesheets
  used in the Flask parts of the website
\item
  Thalia/templates -- Contains Jinja2 templates for different parts of
  the Flask website. Including the base html, homepage, authentication
  pages, different info pages, user portfolio gallery
\item
  Thalia/\_\_init\_\_.py -- Module responsible for the app factory of
  the flask app, passes complete flask apps to web workers such
  gunicorns. Includes flask app setup, configuration, dash app
  registrations and protection, database registrations, extension setup.
\item
  Thalia/extensions.py -- Module for initalising different Flask
  extensions
\item
  Thalia/findb\_conn.py -- Module for accessing finda. Seperated from
  other modules to avoid circular imports
\item
  Thalia/forms.py -- Declarations of different HTML form controllers
\item
  Thalia/views.py -- Main UI logic for Flask code. Handles routing and
  controlling user interactions with the Flask code.
\item
  app.dbexport -- The user data containing database
\item
  config.py -- Flask app configuration declarations, includes
  development defaults incase user has not defined the necessary
  enviromental variables.
\item
  feedback.csv -- user submitted feedback from the website
\item
  main.py -- used as the main method for the project, it provides the
  results of the app factory to be passed to a server
\end{itemize}

\paragraph{Tests}\label{tests-4}

\begin{itemize}

\item
  Tests/Thalia/conftest.py -- setup and teardown code for Flask tests
\item
  Tests/Thalia/test\_factory.py -- test app factory pattern
\item
  Tests/Thalia/test\_user.py -- test user account handling
\item
  Tests/Thalia/test\_views.py -- integration tests for views.py
\end{itemize}

\subsubsection{Selenium tests}\label{selenium-tests}

\begin{itemize}

\item
  Tests/Selenium/contact\_test.py -- test contact form
\item
  Tests/Selenium/dashboard\_test.py -- test dashboard
\item
  Tests/Selenium/login\_test.py -- test login
\item
  Tests/Selenium/main\_test.py -- convenience script to run all selenium
  tests
\item
  Tests/Selenium/navbar\_test.py -- test navigation bar
\item
  Tests/Selenium/README.md -- instructions and info on Selenium tests
\item
  Tests/Selenium/register\_test.py -- registration tests
\item
  Tests/Selenium/social\_test.py -- tests for social media links
\item
  Tests/Selenium/util.py -- utilities used by other selenium tests
\item
  Tests/Selenium/webdriver\_test.py -- test for checking selenium config
  is working
\end{itemize}

\subsubsection{Misc files}\label{misc-files}

\begin{itemize}

\item
  .circleci/config.yml -- defines our continous integrations jobs for
  circle-ci
\item
  .flake8 -- our teams style guide rules for linting python code for the
  project with flake8
\item
  pyproject.toml -- our teams style rules for black, our projects
  automatic code formatter
\item
  pytest.ini -- default rules for pytest to run our projects tests
\item
  README.md -- short instructions on installation, tests and other notes
\item
  requirements.txt -- our projects python requirements with versions
  locked and tested as being compatible with the application
\item
  setup.py -- python setup file with more loose requirements and
  includes requirements necessary for testing and contributing.
\item
  start.sh -- commonly used script for running project in a development
  environment, requires unix shell to work
\item
  .pre-commit-config.yaml -- pre-commit hook for running black before a
  git commit is made to ensure style guide is followed
\end{itemize}

\end{document}