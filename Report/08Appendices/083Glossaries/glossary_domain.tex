\documentclass[main.tex]{subfiles}

\begin{document}

\section{Appendix C - Problem Domain Glossary}
\label{domain_glossary}
\begin{itemize}
    \item Portfolio - A portfolio is a grouping of financial assets such as stocks, bonds, commodities, currencies and cash equivalents, as well as their fund counterparts, including mutual, exchange-traded and closed funds. An investor's portfolio is the group of assets they have currently invested in. [https://www.investopedia.com/terms/p/portfolio.asp]
    \item Asset - Generally an asset that gets its value from being owned; can be traded on financial markets. Stocks, bonds, commodities, (crypto-)currencies are all types of financial assets. \newline[https://www.investopedia.com/terms/a/asset.asp]
    \item Asset Class - An asset class is a grouping of investments that exhibit similar
characteristics and are subject to the same laws and regulations. Asset
classes are made up of instruments which often behave similarly to one
another in the marketplace. \newline[https://www.investopedia.com/terms/a/assetclasses.asp]
    \item Backtesting - Backtesting is the general method for seeing how well a strategy or
model would have done ex-post. Backtesting assesses the viability of a trading strategy by discovering how it would play out using historical data.\newline[https://www.investopedia.com/terms/b/backtesting.asp]
    \item Standard Strategy / Lazy Portfolios - A lazy portfolio is a collection of investments that are designed to perform well in most market conditions and thus require very little maintenance.\newline[https://www.thebalance.com/how-to-build-the-best-lazy-portfolio-2466533]
    \item Rebalancing - Rebalancing is the process of realigning the weightings of a portfolio of assets. Rebalancing involves periodically buying or selling assets in a portfolio to maintain an original or desired level of asset allocation or risk. \newline[https://www.investopedia.com/terms/r/rebalancing.asp]
    \item Key metrics - Performance measures of a portfolio that are of high interest to a majority of investors.
    \item Standard Deviation - The standard deviation is a statistic that measures the dispersion of a dataset relative to its mean.\newline[https://www.investopedia.com/terms/s/standarddeviation.asp]
    \item Worst Year - The worst performance over any given 365 day period starting from January 1st of some year.
    \item Sharpe Ratio - The Sharpe ratio was developed by Nobel laureate William F. Sharpe and is used to help investors understand the return of an investment of a portfolio compared to its risk.\newline[https://www.investopedia.com/terms/s/sharperatio.asp]
\item Sortino Ratio - The Sortino ratio is a variation of the Sharpe ratio that differentiates harmful volatility from total overall volatility by using the asset's standard deviation of negative portfolio returns, called downside deviation.\newline[https://www.investopedia.com/terms/s/sortinoratio.asp]
\item Inflation - Inflation is a quantitative measure of the rate at which the average price level of a basket of selected goods and services in an economy increases over a period of time.\newline[https://www.investopedia.com/terms/i/inflation.asp]
\item Nominal Values - A value that is unadjusted for inflation.
\item Real Values - A value that is adjusted for inflation.
\item Equity - Equity is typically referred to as shareholder equity (also known as shareholders' equity) which represents the amount of money that would be returned to a company’s shareholders if all of the assets were liquidated and all of
the company's debt was paid off.
\item Fixed Income - Fixed income is a type of investment security that pays investors fixed interest payments until its maturity date.\newline[https://www.investopedia.com/terms/f/fixedincome.asp]
\item Commodity - A commodity is a basic good used in commerce that is interchangeable with other commodities of the same type. Commodities are most often used as inputs in the production of other goods or services. The quality of a given commodity may differ slightly, but it is essentially uniform across producers.\newline[https://www.investopedia.com/terms/c/commodity.asp]
\item FOREX / FX - Forex (FX) is the marketplace where various national currencies are traded. The forex market is the largest, most liquid market in the world, with trillions of dollars changing hands every day.\newline[https://www.investopedia.com/terms/f/forex.asp]
\item Overfitting - Overfitting is a modelling error that occurs when a function is too closely fit to a limited set of data points. In the context of investing, a portfolio that performs significantly better on a limited time-frame compared to a larger one is said to be overfit on that timeframe. \newline[https://www.investopedia.com/terms/o/overfitting.asp]
\item Leverage - Leverage results from using borrowed capital as a funding source when
investing to expand the firm's asset base and generate returns on risk
capital.\newline[https://www.investopedia.com/terms/l/leverage.asp]
\item Technical Analysis - Technical analysis is a trading discipline employed to
evaluate investments and identify trading opportunities by analysing
statistical trends gathered from trading activity, such as price
movement and volume. \newline[https://www.investopedia.com/terms/t/technicalanalysis.asp]
\item Drawdown - A drawdown is a prolonged decline during a specific period for an investment, trading account, or fund. A drawdown is usually quoted as the percentage between its peak and susequent trough. \newline[https://www.investopedia.com/terms/d/drawdown.asp]
\end{itemize}

\end{document}
