\documentclass[main.tex]{subfiles}
\ProvidesPackage{preamble}

\usepackage[nottoc]{tocbibind}
\usepackage[english]{babel}
\usepackage[utf8]{inputenc}
\usepackage[table]{xcolor}
\usepackage[nohead, nomarginpar, margin=1in, foot=.25in]{geometry}
\usepackage{tabularx}
\usepackage{graphicx}
\usepackage{float}
\usepackage[english]{babel}
\usepackage{paralist}
\usepackage{datetime}
\usepackage{afterpage}

\begin{document}

\section{Introduction}
\subsection {Project Overview}

The aim of this project was to create a portfolio backtesting software, which enables creating custom portfolios and measuring their performance with different backtesting functions. The user can pick from a variety of assets, some of which are Equities, Fixed Income, Currencies, Commodities, and Cryptocurrencies. Risk metrics and performance are then visualized for the given asset allocation.
\subsection{Motivation/Rationale}
Since retail investing is a growing market, our target audience consists of individual investors, who instead of seeking the full package that comes with financial advising, would like to take over the wheel and assess the viability of their investment strategies themselves. As retail investors are non-professionals and invest comparatively small amounts, with financial advice services being non affordable for those individual clients, our goal was to create a product that would not only be more affordable for small retail investors, but would also include a variety of international assets, which most existing backtesting software fails to provide \cite{WP}.

\subsection{Project management strategy}

The creators of Thalia are:
\begin{itemize}
    \item Martti Aukia,\textit{Team Leader}
    \item Arthur-Louis Heath, \textit{Deputy Team Leader}
	\item Albert Boehm, \textit{Chief Editor}
	\item Marcell Veiner
	\item George Stoian
	\item Daniel Joffe
	\item Weronika Kakavou
\end{itemize}

Our goal was the creation of a high quality industrial prototype of the Thalia backtesting software based on the identified project requirements \cite{TR} and optional features \cite{TR}. The team held regular meetings, both with the project guide, Dr Nigel Beacham, and the course coordinator Dr Ernesto Compatangelo. During the analysis stage meetings took place weekly and were aimed to discuss ideas and requirements. Whereas later, during the implementation stage the team held meeting at least once per week, some of which were aimed to discuss the design for the tool with the inclusion of some coding sessions.

As in the past\cite{TR}, our workflow was centered around the GitHub platform and the tools it provides.
Furthermore, we continued to follow the Egalitarian Team structure, as this worked very well during the first term of the course.
In our Technical Report \cite{TR} we also discussed the use of effort-oriented metrics, i.e. story points, which were assigned to tickets based on the time needed and the functionality of the task.
However, the following term we concluded that in many occasions these metrics failed to successfully measure the size of a task, as they were artificially assigned. Based on this observation, we decided not to make use of them for the rest of the development.

As we were fortunate enough to gain an additional member this term, some of our initial effort was focused on introducing her to the project and team dynamics. Given our access to a large team, it was common to see coding tickets assigned to pairs and groups rather than individuals. We believe this not only resulted in writing better-quality code, but in faster team communication when making design decisions.

\subsection{Budget}
As previously discussed, we did not have any budget restrictions other than time. All of us agreed to allocating a minimum of 10 hours per week to development efforts and discussed personal expectations well in advance.  We estimate the value of our products goodwill to be approximately £13,000. We base this estimate on the developer time allocated to this project, the total duration of 20 weeks, and average developer salaries in the UK\cite{DeveloperSalary} (taking into account that we have no industry experience, meaning the value of our work is realistically towards the lower end of the range).

\end{document}
