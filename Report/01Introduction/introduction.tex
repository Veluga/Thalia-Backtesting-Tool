\documentclass[main.tex]{subfiles}

\begin{document}

\section{Introduction}
\subsection {Project Overview}

The aim of this project was to create a portfolio backtesting tool that allows the creation of custom portfolios and the measurement of their performance on historical data. The user would be able to pick from a variety of assets, including various Equities, Fixed Income Securities, Currencies, Commodities, and Cryptocurrencies. Risk metrics and performance are then calculated, visualized and displayed to the user for analysis. Thalia is our attempt at creating such a tool.
%Thalia as in the name needs referencing before we just list the creators, hence the last line

\subsection{Motivation}
Our target audience is the growing market of retail investment, which consists of individuals who, due to preference or financial limitations, choose to eschew professional money management solutions such as the hiring of financial advisers, and instead take on the task of assessing the viability of their investment strategies themselves. Retail investors tend to be non-professionals who invest comparatively small amounts and have few resources at their disposal. To fill our identified market niche and cater to these investors, we aimed to design a tool that would be accessible, affordable and contain data on a wide variety of financial assets for investors to choose from as this is something current solutions fail to provide \cite{WP}.

\subsection{Project Management Strategy}

The creators of Thalia are:
\begin{itemize}
    \item Martti Aukia, \textit{Team Leader}
	\item Albert Boehm, \textit{Chief Editor}
	\item Arthur-Louis Heath, \textit{Deputy Team Leader}
	\item Daniel Joffe
	\item Weronika Kakavou
	\item George Stoian
	\item Marcell Veiner
	
\end{itemize}

Our goal for this semesters project was the creation of a high quality industrial prototype of the Thalia backtesting tool, based on our identified requirements and optional features \cite{TR}. The team held regular meetings, both with the project guide, Dr. Nigel Beacham, and the course coordinator Dr. Ernesto Compatangelo. During the analysis stage, meetings for the discussion of ideas and the outlining of requirements were held weekly. Later, during the implementation stage, the team held internal meetings at least once per week for the discussion of the tool's design while as well as several joint programming sessions. \newline \newline
%I tried to differentiate here between meetings with staff and internal team meetings so its not just saying meetings we're first held weekly then weekly
As in the past \cite{TR}, our workflow was centred around the GitHub platform and the tools it provides.
Furthermore, we continued to follow the Egalitarian Team structure, as this worked well during the first term of the course.
In the previous terms Technical Report, we discussed the use of effort-oriented metrics, i.e. story points, which were assigned to tickets based on the time needed and the functionality of the task.
However, this term we concluded that in many situations these metrics had failed to accurately measure and predict the size of a task. Based on this observation, we decided to abolish this practice for the rest of the development process. \newline \newline
As we were fortunate enough to gain an additional member this term, some of our initial effort was focused on introducing her to the project and team dynamics. Given our access to a large team, it was common to see coding tickets assigned to pairs and groups rather than individuals. We believe this resulted in better-quality of the codebase and faster team communication and decision making.

\subsection{Budget}
% We had budget restrictions, as we had no budget, we talk about not opting for payed tools for this reason in later sections. Do you mean to say we had no costs?
As previously discussed, we did not have any budget restrictions other than time. The team discussed personal expectations in advance and agreed to allocate a minimum of 10 hours per week to development. We estimate the value of our product's goodwill to be approximately £13,000. We base this estimate on the developer time allocated to this project, the total duration of 20 weeks, and average developer salaries in the UK \cite{DeveloperSalary} (taking into account that we have no industry experience, i.e. that the value of our work is towards the lower end of the range provided).

\end{document}
