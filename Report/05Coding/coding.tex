\documentclass[main.tex]{subfiles}
\ProvidesPackage{preamble}

\usepackage[nottoc]{tocbibind}
\usepackage[english]{babel}
\usepackage[utf8]{inputenc}
\usepackage[table]{xcolor}
\usepackage[nohead, nomarginpar, margin=1in, foot=.25in]{geometry}
\usepackage{tabularx}
\usepackage{graphicx}
\usepackage{float}
\usepackage[english]{babel}
\usepackage{paralist}
\usepackage{datetime}
\usepackage{afterpage}

\usepackage{listings}
\usepackage{xcolor}

\usepackage{dirtytalk}

\begin{document}

\section{Coding and Integration}

After a brief overview of the project and the project plan, this section will focus on the main technologies used in the project and the rationale behind choosing them. Moving on, we will discuss how these components were integrated and eventually deployed.

\subsection{Overview}

One of the first decision we have made this term was to completely recreate our prototype of Thalia. Firstly, this radical move was the consequence of a new implementation decision (for more detail see \ref{Web Framework}). Secondly, as prototypes are meant to be disposable and are designed only to answer key questions about the system \cite{pragmaticprog}, the proof of concept served its purpose, giving us a chance to refine the structure and the quality of the code.

Despite the risks posed by using a Software Version Control (SVC) Host such as GitHub, we have decided to continue using it as our software development platform. The reasoning behind this builds on the argument developed in our Technical Report \cite{TR}, which highlights that our, that our Data Processing Module is a completely separate component in our system which none of the other components is able to access. Additionally, we are storing API keys as environment variables in a file that is not tracked by our SVC system, which minimizes the probability of us exposing sensitive data.



TODO talk about API keys and security measures

We have also decided to develop the application with python as our main choice of programming language. Even though this choice seemed obvious from the beginning, we did consider its main benefits, these would be the following:

\begin{itemize}
    \item Python is a high level programming language allowing us to better focus on the application.
    \item A standard choice for prototyping.
    \item Provides superb third party libraries and frameworks for free.
    \item Easy to integrate if we were to choose other languages at some point in our development.
    \item The whole team was already familiar with the language, saving us the precious time needed to learn another programming language.
    \item Our application does not require an unreasonable amount of computation, so there is no need for a more efficient programming language such as C. \ref{BL} \cite{languagescomparison}.
\end{itemize}

We will discuss other technologies used in more detail after the discussion on project planning.

\subsection{Planning}

Early in the inception phase of development we have decided that our goal was not to have fixed responsibilities in our team, allowing everybody to work on each component of the system. This decision has also eased the code review process, as we had no status differences to distort the error-correcting mechanism \cite{statusdifference}. Furthermore, since no team member was the sole developer of a system component, this allowed us to direct comments at the code and not the author \cite{howtoreview}. For these reasons we have decided to follow the Egalitarian Team structure, and make use of the flexibility offered by it.

Our workflow was centred around the tools provided by GitHub. We used a ticketing system to divide and distribute tasks among team members. These tickets were sometimes given by the team on the weekly meetings, but occasionally they were chosen by the member proposing the feature or change. Our goal with this approach was to divide larger jobs into smaller tasks.

A typical ticket in our project was an encapsulation of a user story, as it consisted of a title, a one liner, value, acceptance criteria, and sub tasks. In the first half of development we also used effort-oriented metrics, called story points to measure the amount of work but we decided to abandon this aspect. An example of a ticket can be seen on \figurename{\ref{Ticket}}.

\begin{figure}[H]
   \centering
   \includegraphics[scale=0.7]{05Coding/05Pictures/ticket.png}
   \caption{Ticket Example (source: https://github.com/)}
   \label{Ticket}
\end{figure}

TODO update git stats

Throughout the whole project we had a total of 110 tickets, and 71 pull requests. The following graph also shows the number of contributions to master, excluding merge commits.

\begin{figure}[H]
   \centering
   \includegraphics[width=\textwidth]{05Coding/05Pictures/contributions.jpg}
   \caption{Contributions to the Master branch (source: https://github.com/)}
\end{figure}

Nevertheless, this was not always achievable especially in the beginning of the development when more crucial components of the systems were developed. In these cases, we assigned the ticket to a pair, or group of people. This approach achieved the following:

\begin{itemize}
    \item Improved the overall code quality and fastened production \cite{pairprogramming}.
    \item Minimised review time on the long run.
    \item Distributed the knowledge of large system components amongst a few people instead of one.
    \item Eased introducing the new team member to the project.
\end{itemize}


In addition, we also had a scrum board as an overview for the ongoing tickets. Although in the Scrum community there are ongoing discussions about the benefits of a physical scrum board over an online one \cite{physicalscrum}, given no actual workplace this was not possible to achieve. This allowed us to see which tickets needed to be reviewed and which were ready to be merged. The tickets/cards were distributed into columns, such as To do, In progress, Review in progress, Review complete and Finished. A truncated picture of this scrum board can be seen on figure \figurename{\ref{Scrum}}.

\begin{figure}[H]
   \centering
   \includegraphics[width=\textwidth]{05Coding/05Pictures/scrumboard.png}
   \caption{Scrum board - truncated}
   \label{Scrum}
\end{figure}

The workflow then largely depended on the task at hand and the person working on it. It was up to them if they met for pair-programming, or chose to work individually. In the beginning we all agreed on the developing environment and coding standards (for a detailed analysis see the later section \ref{Coding Standards}). Since this report was a relevant portion of the workload, we decided to treat it as code. The report was written in \LaTeX, first creating the overall structure of the document with a main.tex compiling the sections together. This let us to work on different sections separately just like features in our software.

Regardless of the nature of the ticket, the week, or in case of some larger tasks, two weeks, ended by the creator indicating that the changes were ready to be integrated. This was done by publishing the changes and creating a new pull request, pushing the changes onto the master branch. To ensure code quality, we set up a continuous integration (CI) environment (more on that in \ref{Coding Standards}). After all checks have passed, the changes were successfully integrated into the code base.

\subsection{Key Implementation Decisions}

Let us now discuss the main technologies used in our project, these can be categorised as follows:
\subsubsection{Web Framework}
\label{Web Framework}
Choosing the right web framework was more controversial than originally planed. The two major options for Python are Django and Flask. Although we decided to use Django for our MVP in the first semester, we had to spend a significant portion of the time available learning the framework, so we had to decide whether we were going to stick with it or learn Flask. In the end, we opted to go with Flask for the following reasons:

\begin{itemize}
    \item Django has one architecture that all projects must share, and we have designed the architecture for our project ourselves. While neither architecture is wrong, the two are not compatible. Flask, on the other hand, is structure-agnostic, so we can lay out the code as we see fit.

    \item Flask comes with the bare minimum for web-development, which means that we don't need to manage the complexity of any feature we're not using. Django has a more complete feature-set from the beginning. This would be desirable in a large web application, but introduces significant overhead in our case, where the website has only a handful of pages.

    \item Django all but insists on using its ORM for all database interaction, while we plan to have a more manual approach.

    \item Our concerns were also confirmed by more experienced web-developers, suggesting simpler alternatives.
\end{itemize}

More precisely, we believe that using such a powerful framework like Django is unnecessary for developing our application. A lot of Django's features wouldn't be used for our type of project. Since Flask is simpler, the entire team thought it would be better to use it for creating a minimalistic app in a short time period. On top of that Flask provides more flexibility, a simple Flask application can be later changed to add more functionality, make it more complex, whereas with Django it is much more complicated. Additionally Django doesn't make it easy to make changes to the already provided modules. Therefore developers create web applications using built-in features which means that if a developer wants to use a different library for a function that Django already provides, it might be hard to do. With Flask it is much easier to make changes to the libraries, databases. 

\iffalse
Sources:
https://www.djangoproject.com/

https://flask.palletsprojects.com/en/1.1.x/

https://www.codementor.io/@garethdwyer/flask-vs-django-why-flask-might-be-better-4xs7mdf8v

http://ddi-dev.com/blog/programming/django-vs-flask-which-better-your-web-app/

https://coderseye.com/django-vs-flask
\fi

For the main framework we chose Flask, a Python-based micro web framework. Flask provides functionality for building web applications, such as managing HTTP requests, using the `requests` library, and rendering templates \cite{smyth_2018}. Flask leverages `Jinja2` as a template engine \cite{templates_2010}.We used the `render-template()` library to process and parse the HTML templates, for all the pages, except for the dashboard, which is described below. We used some of Flask's extensions such as Flask-Login, which provides user session management for Flask \cite{flask-login}. To handle web forms we used the `flask-wtf` extension, which is a wrapper around the `WTForms` package \cite{wtforms-documentation}. 

\subsubsection{GUI}
\label{GUI}

Dash is great for applications that require data visualisation, modelling, or analysis \cite{dash}, which is precisely what is needed for our portfolio backtesting software. It allows us to create reactive single-page apps, meaning that with the use of tables, drop-down menus, etc., the app is updated instantly. This is done with the use of `callbacks`, which whenever an input changes, make an AJAX request whose output value is updated automatically \cite{callbacks}.

A typical callback using an example from our application can be seen below:

\begin{lstlisting}[language=Python, caption=setup.py - Development environment, label=lst:callback_example]
def register_add_portfolio_button(dashapp):
    dashapp.callback(
        Output("portfolios-container", "children"),
        [Input("add-portfolio-btn", "n_clicks")],
        [State("portfolios-container", "children")],
    )(add_portfolio)
\end{lstlisting}

In this code snippet, we register a listener to the button with the unique id: `add-portfolio-btn`, and its property `n\_clicks`, or number of clicks. Whenever the input property changes, the corresponding function gets called automatically. This also happens upon launching the Dash page, therefore it is common to start the called function by catering to this case, and raising a `PreventUpdate` exception if the input parameter is `None`. An example can be seen in \cite{prevent_update}.

The called function then receives the input properties as parameters. Furthermore, we can add any number of parameters as `State` objects, as has been done above. When the function finishes, it needs to return the same number of values as is specified in the callback. Mismatching the number of return values results in a runtime error. Whereas mismatching the input parameters yields a different function signature, and so the callback is not registered to the desired function, doing so does not raise any errors. 

Fortunately, Dash offers great developer tools \cite{dash_dev_tools}, such as the `Callback Graph` visible on \figurename{\ref{callback_graph}}, which can be accessed by running the application in debug mode and activating the Dev tools. This graph came in handy often, especially due to the arguments presented in \ref{dash_limitations}.

 \begin{figure}[H]
   \centering
   \includegraphics[scale=0.6]{05Coding/05Pictures/callback_graph.png}
   \caption{Callback Graph on Thalia}
   \label{callback_graph}
\end{figure}


\subsubsection*{Limitations}
\label{dash_limitations}

Although Dash is fairly intuitive and great for one-page applications, it struggles when it comes to dynamically created components. To be more precise, all callbacks need to be registered before runtime. For example, in our application the user is able to enter multiple portfolios via the `Add Portfolio` button. Doing so creates new input fields for the portfolio which all need to be registered. 

As this problem persisted with other components as well, we needed to find some solutions. Many approaches have been suggested by the Dash Community, some even by the developer team \cite{dash_workaround}. At the end, we have decided to use the most stable approach and generate all the components beforehand and hide them. This means that the relevant function now also has the responsibility of returning the visibility of the required Div. Note that this is a highly requested feature in the Dash community, and is likely to be added soon, but for now, we needed to resort to workarounds.

Furthermore, the data we show in the dashboard is greatly centered around the strategy object discussed in \ref{BL Structure}. At the beginning of every backtesting process a new strategy object is initialised, which is a time-costly task. As this object holds most of the information our result pages require, many functions would need to access the same object. 

One solution would be to have a callback for initialising the object, and then share it with all others. Although there exist ways to do so \cite{dash_share}, we were not satisfied with any of the approaches listed. Consequently, we decided to go forward with the original approach that Dash favours, and combine more outputs in one function. Even though this introduced more complexity to our code, it made the application faster overall.

\subsubsection*{Layout}

The layout of the application, as discussed in \ref{GUI}, has been created with the use of the `Dash HTML components` library. Consequently, there was no need to write HTML for the dashboard, since the layout was composed using Python, which generates HTML content. Dash uses the Flask web framework, which makes it fairly simple to embed a Dash app at a specific route of an existing Flask app \cite{flask-dash}. The framework also handles most of the communications between the front-end and back-end.

The User-Interface was made with the use of `Bulma`, which is an open-source CSS framework \cite{bulma}. Bulma is ideal for creating responsive elements as it automatically adapts the website to different screen sizes \cite{responsiveness_2020}. It is also compatible with various web browsers. Another advantage of Bulma is that we could create a very modern-looking website fairly simply as the framework is doing a lot of styling for us.

It also provides pre-styled elements, components, such as the navigation bar or alerts for the users. It is fairly simple to learn in comparison to `Bootstrap`, as Bulma has easy to learn syntax, it has simple readable class names like `.button` and has a very straightforward modifiers system, for example,  `.is-primary`. There are no JavaScript elements, only CSS, which makes it adaptable to frameworks such as Dash.


\subsubsection{Business Logic}
\label{BL}

The business logic module sits at the very core of our application. We require it for producing a time series of a portfolio's performance as well as selected key risk metrics of a given asset allocation. This output is consumed by our web application and presented in plots and tables as described in \ref{GUI}.

Research into developing this module was initiated by listing our requirements for its desired behaviour. We identified that it should support:

\begin{itemize}
    \item Specification of a portfolio as a set of pandas dataframes, the common data exchange format in our application
    \item Calculation of a portfolio's return over time
    \item Regular contributions to mimic saving
    \item Rebalancing strategies to reestablish the desired weighting of a portfolio
    \item Calculation of key metrics, including the Sharpe and Sortino Ratio, Max Drawdown, Best and Worst Year
    \item Collecting and reinvesting dividends for equities
\end{itemize}

Using these requirements, we struggled to find an open-source library that would handle these tasks for us. While backtesting libraries written in Python are available in abundance (e.g. PyAlgoTrade \cite{PyAlgoTrade} and bt \cite{bt}), each of these was lacking in at least one critical aspect. None of them support specifying a portfolio using absolute or relative weights and instead seem to focus on backtesting trading strategies involving just a single asset while using technical indicators. Thus, we made the decision to develop our own library for handling the aforementioned tasks.

The result of this effort is `Anda` (short for analyse data). For each of its functions, Anda takes as input a Strategy object that specifies the entire list of parameters for a backtest, including contribution dates, a list of assets with associated price data, dividends for equities, etc. Calculations are performed on a per-metric basis by separating them into individual functions.
This approach has allowed us to tailor the entire business logic module exactly to our needs without having to produce complicated wrapper code for existing backtesting libraries.

\subsubsection{Database \& Finda}
\label{Finda}

Another major technology decision was the choice of appropriate database management system (DBMS) for storing historical price data collected by the data harvester. Before committing to a specific technology we identified the following requirements a suitable DBMS should fulfil:

\begin {itemize}
\item \textbf{Schema:} The structure of our data is relatively simple, consequently Thalia does not require support for sophisticated features and data types. A suitable DBMS should be able to accommodate the database schema designed last term, with the addition of simple integrity constraints and cascade operations.
\item \textbf{Support:} Ideally the DBMS should be cross platform, as this would allow us to defer commitment to a specific deployment platform until we are ready to start the CD process. 
\item \textbf{License and pricing:} The DBMS should be free to use and have a non-restrictive license.
PERFORMANCE: The DBMS should be able to handle a high volume of concurrent reads to fulfil user requests. The data will be updated daily, meaning efficient write operations are a lower priority.
\item \textbf{Usability:} As our team lacks experience in this field, a suitable DBMS should be relatively simple to learn. Ideally, team members should be able to learn the basics in a single weekly sprint.
\item \textbf{Security:} The DBMS should have a mature code base and be relatively secure, as access to financial data is a key component of our business model. Later it will likely also store data that is not available through public APIs, meaning potential data breaches could expose us to legal liability. \cite{dataprotectionGov}
\item \textbf{Type:} Since the project constraints specify we use SQL queries, only relational DBMS supporting a version of SQL are appropriate.

\end{itemize}

MySQL, PostgreSQL, SQLite and MariaDB were subject to in depth comparison based on fulfilment of the above requirements and industry adoption \cite{dbPerfComparison} \cite{dbmsMarketShare}. Our final decision was to use SQLite for the following reasons:

It is user-friendly and easy to deploy, allowing us to start continuous deployment faster.
It has a small footprint and offers good performance. \cite{dbPerfComparison}
Portable serverless design aids with development and testing.
All team members have experience working with SQLite from previous term. This helps to reduce the overhead of knowledge transfer.

The main drawbacks of using SQLite, namely scalability and performance are not a concern at this stage, as the current version of Thalia is meant to be a high-quality industrial prototype, and as such will not contain the full range of financial data needed for marketability. Should SQLite prove to be inadequate in the future, we would be able to switch to a different DBMS with relatively little trouble, as the process of database migration is exceedingly well-documented \cite{frameworkDataMigration} \cite{understandingDataMigration}. To pre-empt any difficulties that might arise, the decision was made to design the data layer to easily accommodate such a migration.



\subsubsection{Data Harvester}
\label{Data Harvester}
The Data Harvester role is to provide the application with the data it requires in order to allow the users to construct their investment portfolios. The choices we have made had to fulfil all our requirements while keeping in mind our financial and time-related limitations.

\subsubsection*{Chosing the  APIs}
We started our quest for data by trying to find APIs that could call for the main financial assets used when making an investment portfolio \cite{best-stock-api}.
We have seen that there were several APIs that could be used. After a quick look at them, we have seen that there were two types of APIs. There were those that had FOREX \cite{forex} data and those designed for Stock Market Data \cite{stock_market}. Those two APIs that we considered for FOREX data were Fixerio and Nomics. For stock market data we have considered yfinance, Alpha Vantage and Quandl.\newline


\subsubsection*{FOREX API}
Between nomics \cite{nomics-python} and fixerio \cite{fixerio} we have chosen nomics for the simple fact that nomics had unlimited calls for free, besides, nomics also has cryptocurrencies data. However, Fixerio is a more robust API with a larger user base and more data, that we would have used if could have afforded the investment. In the scenario where we would have had money to spend we would have used fixerio for currencies and nomics for cryptocurrencies. However, in our case we have no money so we used nomics for both currencies and cryptocurrencies. To compare the two options we have used the documentation provided by each API . \newline

\subsubsection*{Stock Market Data}
Out of yfinance \cite{yfinance}, Alpha Vantage \cite{alphavantage}and Quandl \cite{quandl} we have chosen yfinance because it offered the most data and API calls for free, compared to the other two. If we had money to spend we would have used Alpha Vantage because of the larger number of available assets and the high number of requests per minute. As for Quandl we found it to be expensive and not centred on the type of data we need. We have made our choice based on the API specification given in the documentation of each API.\newline

\subsubsection*{Standard Data Format}
While looking for solutions that would help with solving the data format and the maintenance problem we have found pandas\_datareader \cite{pandas_datareader}. It is a wrapper for the biggest financial APIs. We saw that it included all of the Stock Market Data APIs named above and more. 
In addition, it standardizes the calling procedure for all APIs it includes and it returns the same format no matter the API you are calling. This is the only product that does this on the market so no compassion with other product can be done. In the case where we would have not used it we would have produced code that would have archived the same result. In conclusion, using pandas\_datareader has saved us some valuable man-hours.\newline
    
\subsubsection*{Update the database \& Redundancy \& Data Interpolation}   
For these requirements, we have seen that no already existing systems can do what we want. As a result, we had to write our system.
In the building of the system, we have used python as a programming language. This decision has been made because the APIs chosen have been built in python. Writing the system in a different language while the APIs are in python would have created additional problems without any benefits.  To manipulate the data we have used the pandas library. The persistent data required by the update mechanism is stored in CSV \cite{csv_iso} format because of the readability and the ease of use when using pandas.

\subsection{Integration and Deployment}
\label{Coding Standards}

 Writing well documented and good quality code is one thing, but making sure it all works together as a whole is a completely different story. In the first term, many hours have been wasted on trying to integrate different components of the system which did not want to fit together. Even then we had some DevOps tools in place \cite{DevOps}, but considering that we had to produce significantly less code and more documentation last term, this was not a priority.

 Starting afresh the coming term, we have decided to set up the development environment again. Upon opening the setup.py file, we see a list of required libraries, and the following two lines of code:

\begin{lstlisting}[language=Python, caption=setup.py - Development environment, label=lst:Development_env]
"""A setuptools based setup module."""
from os import path

from setuptools import find_packages, setup

here = path.abspath(path.dirname(__file__))

install_requires = [
    "flask",
    "flask-login >= 0.5",
    "flask-migrate",
    "flask-wtf",
    "pandas",
    "dash",
]


tests_require = ["pytest", "coverage"]

extras_require = {"dev": ["black", "flake8", "pre-commit"], "test": tests_require}

setup(
    name="Thalia",
    version="0.2.0",
    packages=find_packages(),
    install_requires=install_requires,
    extras_require=extras_require,
)
\end{lstlisting}

One of the first decisions we had to make, is to decide on a standards coding style. This is exactly what flake8 is for, which we can see amongst the extras in the code snippet above. The original documentation of flake8 defines it as " [...] is a command-line utility for enforcing style consistency across Python projects. By default it includes lint checks provided by the PyFlakes project, PEP-0008 inspired style checks provided by the PyCodeStyle project, and McCabe complexity checking provided by the McCabe project" \cite{flake8}. However, as many other developers we also decided to redefine the maximum line-length from 79 to 88 as we found this convention a hindrance.

Another tool used for enforcing uniform style was the black auto-formatter for Python \cite{black}, which formatted the code for us upon every save if enabled, and also when commiting code. This has been achieved by the use of githooks \cite{githooks}, which are programs that are triggered upon certain git actions. For this we needed the pre-commit package for setting up these actions and writing the configuration file. This ensured that both flake8 and black have been run before publishing changes.

\subsubsection{Continuous Integration}
\label{Continuous Integration}

Many studies have investigated the positive effects of developing in a continuous integration (CI) environment \cite{CI_1}, \cite{CI_2}. Regardless of the exact implementation, its obvious benefit is that it provides security and uniformity for projects. We already made some steps to achieve a uniform style, but had no means to know whether the code published has been also passed its tests. Note in this section we will only discuss testing as a part of CI and not the testing strategy, for that see the section \ref{Testing}.

Another important part of the DevOps toolchain is the use of containers, which is what Docker helps to achieve \cite{Docker}. Docker helps developers focus on writing code rather than worrying about the system the application will be running on, and also helps to reveal dependency and library issues. As Docker is open source, there are many free to use docker images available online \cite{DockerImages}. When choosing the CI environment, Docker support was one of the main requirements.

The most promising candidate for this was CircleCI \cite{CircleCI}, which is a cloud-based system with first-class Docker support and a free trial. After connecting our GitHub repository to CircleCI, and setting up a configuration, CircleCI now does the following on every pull-request:

\begin{enumerate}
    \item Sets up a clean container or virtual machine for the code.
    \item Checks previously cached requirements, for more detail see \figurename{\ref{cache_CI}}.
    \item Installs the requirements from requirements.txt
    \item Caches the requirements for faster performance.
    \item Clones the branch needed to be merged.
    \item Runs flake8 one last time and saves results.
    \item Runs tests and saves results.
    \item Deploys the master branch to AWS, see \ref{Continuous Deployment}
\end{enumerate}

 \begin{figure}[H]
   \centering
   \includegraphics[width=\textwidth]{05Coding/05Pictures/cache_CI.png}
   \caption{CircleCI on Caching (source: https://circleci.com/docs/2.0/caching/)}
   \label{cache_CI}
\end{figure}

The outcome of these steps is visible on CircleCI, but more importantly also on GitHub, and it refuses to merge if failing test (or no tests) have been found.

 \begin{figure}[H]
   \centering
   \includegraphics[scale=0.6]{05Coding/05Pictures/circleCI.png}
   \caption{CircleCI on GitHub (source: https://github.com/)}
   \label{CircleCI}
\end{figure}

With the help of these steps and CircleCI we managed to ensure that the code written is uniform and of good quality. It significantly reduced the time needed for integrating and code reviewing. The last step was now to deploy the system.

\subsubsection{Hosting and Continuous Deployment}
\label{Continuous Deployment}

An overview over the literature covering reveals a plethora of different strategies for deploying and hosting web applicaitons \cite{ConnollyFundamentals}. Our decision of how to choose among them involved the following considerations:

\begin{itemize}
    \item Price - since we have severe budget constraints, we were looking for a cheap hosting solution
    \item CircleCI support - The target host should be supported by CircleCI natively to ease development of a continuous deployment (CD) pipeline
\end{itemize}

The upfront cost of buying physical servers ruled it out as an option for us. Thus, we turned our attention to using a solution that involved deployment to a virtual machine in the Cloud.

Due to native CircleCI support and a free-tier service, we initially chose Heroku \cite{Heroku} as our initial hosting provider. This allowed us to host our application for free in the initial stages of development while providing ample opportunity for horizontal and vertical scaling later on, if required.

However, it quickly became apparent that Heroku was not appropriate for our purposes, due to its lack of support for sqlite databases \cite{HerokuSqlite}. Hence, we moved to AWS \cite{AWS}, which is also supported by CircleCI \cite{AWS}.

The benefits of using continuous deployment have been well established for multiple years and involve ``the ability to get faster feedback, the ability to deploy more often to keep customers satisfied, and improved quality and productivity`` \cite{CDBenefits}. Using AWS in combination with CircleCI, our CD pipeline involves the following simple steps:

\begin{enumerate}
    \item Upon commits to the master branch on GitHub, CircleCI triggers a workflow.
    \item The workflow first executes the steps listed in \ref{Continuous Integration} to ensure the validity of the current codebase state.
    \item If this step is successful, the master branch is pushed to a remote repository recognized by AWS via git.
    \item AWS executes the Procfile script stored in the root of our project to start the application using a gunicorn web server \cite{Gunicorn}.
\end{enumerate}

Our deployment process is thus fully automized and immune to failing tests, as it will only complete successfully if the application is in a correct state. The full CI/CD workflow is captured by the flowchart on \figurename{\ref{CI_CD}}.

 \begin{figure}[H]
   \centering
   \includegraphics[scale=0.6]{05Coding/05Pictures/CI_CD.png}
   \caption{CI/CD workflow}
   \label{CI_CD}
\end{figure}

\end{document}
