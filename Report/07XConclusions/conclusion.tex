\documentclass[main.tex]{subfiles}

% Discussion of outcome of project, if original client requirements were not satisfied - explanation of why, possible future changes/additions to the system, lessons learnt. 

\begin{document}

\section{Conclusion and Further Work}

\subsection{Conclusion}

In this report, we have presented Thalia, a portfolio backtesting web application. After engaging in discussion of the product requirements and our approach to development, we have provided justifications for any design and implementation decisions as well as technology choices. In the following sections, we then went on to specifying our testing strategy and the results of our evaluation process. \newline
As a consequence of the preceding evaluation section, we perceive our project to have been successful. We have delivered a high quality, functioning piece of software that fits our identified business needs and market niche. Since we have already provided extensive discussion of whether our requirements have been met, we may now provide an outline of further work that may be done in the future.

\subsection{Further Work}

\subsubsection{Payment Service Integration}

Charging for the service we provide requires processing of customer payments. Given the sensitive nature of payment data (e.g. credit card numbers), our preferred approach is to outsource processing of payments to a third-party service, such as Stripe \cite{stripe}. Our responsibility would then be reduced to integrating the vendor payment system into our codebase.

\subsubsection{Implementation of a Scripting Language}

One of our optional features from last semester's technical report was the development of a scripting language that would allow investors to specify dynamic trading strategies. As an example, an investor may want to sell their equities for bonds whenever there is a drawdown greater than 10\% in equity price indices. While support for such a scripting language was outside of the scope of this project, it is technically feasible, as has been demonstrated by other backtesting tools, e.g. Stockbacktest \cite{stockbacktest}.

\subsubsection{Additional Risk Metrics}

Our aim for this project was to provide values for the most commonly encountered risk metrics when backtesting a portfolio. However, many other, more exotic risk metrics exist. Over time, we may want to incorporate additional metrics into our dashboard to enable an even more nuanced analysis of an investment strategy. To prevent more novice users from being overwhelmed, these could be deactivated by default and enabled on demand.

\subsubsection{Expanding Asset Selection}

Our team is proud of the breadth of assets available for analysis in our tool. As discussed previously, the inclusion of European market assets sets us apart from many of the commercially available tools. Nevertheless, we were unable to support every asset class and even every asset within the asset classes we support. In particular, we would be interested in supporting standard exchange traded derivatives, such as Futures and Options. In order to appeal to other investor demographics, it may also be worthwhile integrating price data for African, Asian and Latin American assets.

\subsubsection{Equity Dividends}

Investors receiving dividends for the equities they own face a choice: Should dividends be reinvested in the equities themselves or spread over the whole portfolio? Supporting calculations that involve dividends would enable us to provide empiric answers to these questions. Our business logic library already provides support for dividend calculations, hence this would be a simple matter of making appropriate changes to the UI.

\end{document}
