\documentclass[main.tex]{subfiles}

\begin{document}

\section{Retrospection}\label{retro}

\subsubsection{Planned schedule}\label{planned-schedule}

Our original schedule was focussed on a few features we decided where
high priority during our fall planning. We placed a large focus on the
core backtesting features we found most important to compete with
existing backtesting tools on the market. As can be seen in the figure
\ref{Roadmap}, our schedule had the core backtesting features done by
the end of week 4, with the following weeks being additional features
and the final weeks including features though of as nice to have but not
core to the project.

\subsubsection{Final schedule}\label{final-schedule}

Our first week was spent mostly on meetings and discussion on how to
start work on the prototype. The following 4 weeks followed the original
schedule quite closely. We started working on the basic implementations
of all our core modules. In addition to working on coding, we had the
team spend time on getting familiar with different collaboration tools
and the process of working with a team this big.

The next 4 weeks were spent on integrating and polishing the modules to
have a consistent code quality, documentation and test coverage across
the project.

The final stretch of the project was spent on additional features and
bug fixes. All modules had at least a couple new features added during
this period, with the UI receiving the most effort. There were also some
new features not part of the original schedule but came up during user
testing as of importance. Including overfitting warnings and
user-uploaded CSVs.

\subsubsection{Most important
differences}\label{most-important-differences}

All the features except for exporting PDFs of the backtest results where
accomplished and on time. As we use an agile approach, we found the
overfitting and user-uploaded CSVs were more important to users than PDF
generation.

Our schedule for the different stages of development: MVP and release
1.0 were overdue by about a week as the first week of the project was
mostly spent on self-organizing and familiarising the team with working
as a group. We completed the MVP for each section by week 5 but only
after week 7 was the project integrated into a single application, with
a nicely polished version done by week 9.

Although all but one feature was completed from the original schedule
new tasks such as work on the financial database manager, additional
features to the business logic and user interfaces features, in general,
took more time and where often a large concern than some of original
schedules features.

It should also be mentioned that great amount of time was spent on
report work during the entirety of our schedule and during the last
weeks, we had our team working on making a high-quality video
presentation for the product.

\subsubsection{Causes for delay}\label{causes-for-delay}

Our delays have to do with matters both internal and external.

Our schedule was at times late by a week due to spending quite a bit of
time learning to work together and set up our work processes. As we had
a large team, learning to parallelise and integrate work was a big
challenge.

Additionally, we did have a 3-week extension to the deadline as some
user-facing features we wanted to add were missing and the team
struggling with adjusting to the COVID-19 pandemic.

\subsubsection{Teams feelings on the
project}\label{teams-feelings-on-the-project}

As a team, we were both demanding of high-quality work and forgiving of
unexpected problems. We didn't have any large scale setbacks but we were
surprised at times at the amount of work necessary to satisfy our high
standard for quality.

It does feel bad at times to leave work in the backlog but as we do use
the Agile method we need to prioritise the most important work and
requirements are rarely static.

Some minor things such as problems with specific technologies
limitations, issues with scheduling meetings with a team of this size
and at times, were issues. Overall though, the team is proud of our
project and happy with working together. At times our high expectations
for ourselves did tire people, but on the other hand, we had each
other's backs and getting help was quick and easy.

\subsubsection{What would we have done
differently}\label{what-would-we-have-done-differently}

If we did again, we would have scheduled less work for the first week as
readjusting to work after a holiday does take time.

Ideally, we would be more comfortable with our technology stacks next
time. We are a team of new developers and a survey of things people
learned during the project found everyone had long lists.

\section{Conclusion and Further Work}

%I think it would be nicer if you wax poetic about the overall project being a sucess after the future section. This way the document just ends witout a proper outro
% Spcicially tale the first paragrap and put it at the end, and make the second paragraph an intro to the Further Work section

\subsection{Conclusion}

In this report, we have presented Thalia, a portfolio backtesting web application. After engaging in discussion of the product requirements and our approach to development, we have provided justifications for any design and implementation decisions as well as technology choices. In the following sections, we then went on to specify our testing strategy and the results of our evaluation process. \newline
As a consequence of the preceding evaluation section, we perceive our project to have been successful. We have delivered a high quality, functioning piece of software that fits our identified business needs and market niche. Since we have already provided extensive discussion of weather our requirements have been met, we will now provide an outline of further work to be done moving forward.

\subsection{Further Work}

\subsubsection{Payment Service Integration}

Charging for the service we provide requires the processing of customer payments. Given the sensitive nature of payment data (e.g. credit card numbers), our preferred approach is to outsource processing of payments to a third-party service, such as Stripe \cite{stripe}. Our responsibility would then be reduced to that of integrating the vendor payment system into our codebase.

\subsubsection{Implementation of a Scripting Language}

One of our optional features from last semester's technical report was the development of a scripting language that would allow investors to specify dynamic trading strategies. As an example, an investor may want to sell their equities for bonds whenever there is a drawdown greater than 10\% in equity price indices. While support for such a scripting language was outside of the scope of this project, it is technically feasible, as has been demonstrated by other backtesting tools, e.g. Stockbacktest \cite{stockbacktest}.

\subsubsection{Additional Risk Metrics}

Our aim for this project was to provide values for the most commonly encountered risk metrics when backtesting a portfolio. However, many other, more exotic risk metrics exist. Over time, we may want to incorporate additional metrics into our dashboard to enable an even more nuanced analysis of an investment strategy. To prevent more novice users from being overwhelmed, these could be deactivated by default and enabled on demand.

\subsubsection{Expanding Asset Selection}

Our team is proud of the breadth of assets available for analysis provided by our tool. As discussed previously, the inclusion of European market assets sets us apart from most other commercially available tools. Nevertheless, we were unable to support every asset class and even every asset within the asset classes we support. In particular, we would be interested in supporting standard exchange traded derivatives, such as Futures and Options. In order to appeal to other investor demographics, it may also be worthwhile integrating price data for African, Asian and Latin American assets.

\subsubsection{Equity Dividends}

Investors receiving dividends for the equities they own face a choice: Should dividends be reinvested in the equities themselves or spread over the whole portfolio? Supporting calculations that involve dividends would enable us to provide empiric answers to these questions. Our business logic library already provides support for dividend calculations, hence this would be a simple matter of making appropriate changes to the UI.

\end{document}
