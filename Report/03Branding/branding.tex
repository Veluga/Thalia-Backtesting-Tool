\documentclass[main.tex]{subfiles}
\ProvidesPackage{preamble}

\usepackage[nottoc]{tocbibind}
\usepackage[english]{babel}
\usepackage[utf8]{inputenc}
\usepackage[table]{xcolor}
\usepackage[nohead, nomarginpar, margin=1in, foot=.25in]{geometry}
\usepackage{tabularx}
\usepackage{graphicx}
\usepackage{float}
\usepackage[english]{babel}
\usepackage{paralist}
\usepackage{datetime}
\usepackage{afterpage}

\usepackage{listings}
\usepackage{xcolor}

\usepackage{dirtytalk}

\begin{document}

\section{Branding}

\subsection{Name and Logo}
In order to find a name and a logo that would represent us, we have done several brainstorming sessions where we have laid out the keywords that represent our product. Out of all the words we listed, two recurring categories appeared, which were flourishing and data. We then went on to find ways of representing those. At the end, we have decided on choosing Thalia as the name of the product, which is the name of a Greek muse often described as the growing one. We saw the association between our product and the Greek muse of growth beneficial.

To create a logo to our liking, we have asked a friend skilled in design to create that Pro Bono for us. After a few iterations, we had two logos made from copyright free images, the smaller of which can be seen on \figurename{\ref{small_logo}}.

\begin{figure}[H]
    \centering
    \includegraphics[scale=0.4]{03Branding/Pictures/small_logo.png}
    \caption{Thalia Logo - Small}
    \label{small_logo}
\end{figure}

\subsection{Motto}

Based on our previous findings in \ref{reliability}, we have decided that our motto should be something that would reassure our customers that they have made the right choice. As a result, we have come up with a simple motto, `Thalia the reliable Backtester`.


\subsection{Official colours} 

The idea of defining a set of official colours came from homogeneity. We strove for a uniform design both for our product and this report, so we believed it was a good way of ensuring this. Given our team's limited experience with design, we have decided to start by choosing the base colour of our palette. In order to be up to date with the latest design trends, we have sought to use something close to the colour of the year 2020 \cite{pantone}. Based on this colour we have created three colour palettes, using a triadic colour scheme \cite{triadic}.

After some deliberation inside the team and some feedback from a few outside people, we have decided to use the third option. In the following figure, the final three colour schemes are presented with the chosen one named Thalia and first one looking left to right.

\begin{figure}[H]
    \includegraphics[width=\textwidth]{03Branding/Pictures/color_schemes.png}
    \caption{Color schemes}
\end{figure}

\end{document}    