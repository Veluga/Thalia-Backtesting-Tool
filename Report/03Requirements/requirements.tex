\documentclass[main.tex]{subfiles}
\ProvidesPackage{preamble}

\usepackage[nottoc]{tocbibind}
\usepackage[english]{babel}
\usepackage[utf8]{inputenc}
\usepackage[table]{xcolor}
\usepackage[nohead, nomarginpar, margin=1in, foot=.25in]{geometry}
\usepackage{tabularx}
\usepackage{graphicx}
\usepackage{float}
\usepackage[english]{babel}
\usepackage{paralist}
\usepackage{datetime}
\usepackage{afterpage}

\begin{document}

\section{Requirements}
Earlier in the inception phase of the development, we classified our requirements using the established FURPS+ model \cite{FURPS}. Since identifying requirements for our initial project report \cite{TR_REQS} we have also identified some additional requirements that Thalia should fulfil based on feedback from our project guide. Let us briefly revisit our main functional and non-functional requirements. The items outlined below focus on some of the key requirements we have so far identified as features necessary for providing a compelling product for paying customers.

\subsection{Functional Requirements}
 
{
\setlength{\tabcolsep}{30pt}
\renewcommand{\arraystretch}{2}
\centering
\rowcolors{2}{gray!25}{white}
\begin{tabularx}{\linewidth}{|X|X|}
\hline
 \textbf{Portfolio Configuration}  &  \\
 \hline
 Allocate fixed amount/proportions of the portfolio to given assets & Choose how much each asset contributes to the portfolio's total value using either percentages or raw monetary amounts \\
\hline
Find assets quickly by category or name & When adding an asset the user can search a category for assets or search for a specific asset by its name \\
\hline
Share portfolio & Portfolios can be shared between people using a URL \\
\hline
Edit portfolio & Change asset allocation and their distributions in a portfolio \\
\hline
\end{tabularx}
}


{
\setlength{\tabcolsep}{30pt}
\renewcommand{\arraystretch}{2}
\centering
\rowcolors{2}{gray!25}{white}
\begin{tabularx}{\linewidth}{|X|X|}
\hline
 \textbf{Portfolio analysis}  &  \\
 \hline
 Compare portfolios & Use multiple portfolios in a single analysis to see differences in their performance \\
\hline
Use a selection of  lazy portfolios & Select an existing common portfolio to compare against, such as common index funds (e.g. Vanguard 500 Index Investor or SPY) \\
\hline
Plot portfolio as a time-series & View portfolio performance as a line graph for a quick overview \\
\hline
Specify a time frame for the analysis & Select start and end dates for portfolio analysis \\
\hline
Choose rebalancing strategy & Optionally choose a strategy for buying and selling assets to meet your strategy e.g. buying and selling stocks each year to ensure the value of portfolio stays at 60\% stocks and 40\% bonds (i.e. maintain the initial allocation) \\
\hline
Change the distribution of assets in a portfolio using a slider & A slider for each asset to quickly increase or decrease its proportion of the total value \\
\hline
Edit portfolio analysis & Change parameters for portfolio's analysis after running it (e.g. date range or rebalancing strategy) \\
\hline
\end{tabularx}
}

\vspace{0.5cm}

{
\setlength{\tabcolsep}{30pt}
\renewcommand{\arraystretch}{2}
\centering
\rowcolors{2}{gray!25}{white}
\begin{tabularx}{\linewidth}{|X|X|}
\hline
 \textbf{View results}  &  \\
 \hline
 See key numerical figures & Show important numerical metrics for a portfolio's performance such as Initial Balance, Standard Deviation, Worst Year, Sharpe Ratio, and Sortino Ratio \\
\hline
See both real and nominal values & See portfolio's value as both adjusted and not adjusted for inflation \\
\hline
A breakdown of portfolio value at specific points of time & See what the value of the portfolio is at some point in time (e.g. January 3rd 1997) \\
\hline
Export result of analysis & Exports results to PDF for sharing and offline reading \\
\hline
\end{tabularx}
}


{
\setlength{\tabcolsep}{30pt}
\renewcommand{\arraystretch}{2}
\centering
\rowcolors{2}{gray!25}{white}
\begin{tabularx}{\linewidth}{|X|X|}
\hline
 \textbf{User accounts}  &  \\
 \hline
 Combine portfolios & Combine two portfolios' assets into one single portfolio \\
\hline
Save portfolio analysis for later & Save portfolio analysis parameters to the account so you can rerun it with a single click \\
\hline
Delete saved portfolio analysis & Remove a stored portfolio analysis from your account \\
\hline
Manage portfolio analyses & Edit saved portfolio analysis with different assets, distributions or other parameters \\
\hline
Sign-up, log in and log out & Basic authentication \\
\hline
\end{tabularx}
}

\vspace{0.5cm}

{
\setlength{\tabcolsep}{30pt}
\renewcommand{\arraystretch}{2}
\centering
\rowcolors{2}{gray!25}{white}
\begin{tabularx}{\linewidth}{|X|X|}
\hline
 \textbf{Assets}  &  \\
 \hline
 Choose assets from European market & Data for European assets were found to be lacking in competing products \\
\hline
Choose assets from Equities, Fixed Income, Currencies, Commodities, and Cryptocurrencies & Coverage of some of the largest asset classes \\
\hline
\end{tabularx}
}

\subsection{Non-functional Requirements}

\begin{enumerate}
   \item Usability:
   \begin{itemize}
     \item The product must be easily usable for users who already have some financial investment experience.
     \item The basic backtesting interface needs to look familiar to people already experienced with it.
     \item The product must have detailed instructions on how to use its advertised functions.
     \item All major functions must be visible from the initial landing page.
     \item Must work in both desktop and mobile browsers.
     \item The results page should scale with mobile.
   \end{itemize}
   \item Reliability:
      \begin{itemize}
     \item The product must have a greater than 99\% uptime.
     \item All our assets need to have up to date daily data where the asset is still publicly tradeable.
     \item All assets supported by the system must provide all publicly available historical data.
   \end{itemize}
    \item Performance:
      \begin{itemize}
     \item The website should load within 3 seconds on mobile [2].
     \item Large portfolios must be supported - up to 300 different assets.
   \end{itemize}
    \item Implementation:
      \begin{itemize}
     \item The system needs to work on a cloud hosting provider.
   \end{itemize}
    \item Interfacing:
      \begin{itemize}
     \item The Data Gathering Module must never use APIs stated to-be-deprecated within a month.
     \item The Data Gathering Module must not exceed its contractual usage limits.
   \end{itemize}
       \item Operations:
      \begin{itemize}
     \item An administrator on-call will be necessary for unexpected issues.
   \end{itemize}
          \item Packaging:
      \begin{itemize}
     \item The product needs to work inside a Linux container (e.g. Docker).
     \item All dependencies need to be installable with a single command.
   \end{itemize}
             \item Legal:
      \begin{itemize}
     \item All user testing must be done with ethical approval from the University.
     \item UI must display a clear legal disclaimer about the service not providing financial advice.
     \item All third-party code should allow for commercial use without requiring source disclosure (e.g. no GPL-3).
     \item User data handling should comply with GDPR.
     \item Provided services should not constitute financial advice under UK law to avoid being subject to financial advice legislation and potential liability
   \end{itemize}
        \item Accessibility
       \begin{itemize}
   \item Display items should be clearly labelled 
   \item UI should scale to accommodate different screen sizes and aspect ratios
   \item UI elements and text superimposed over one another should have high contrast in their colors
   \item UI should allow for the use of assistive technologies to accommodate individuals with accessibility issues
       \end{itemize}
\end{enumerate}

In the literature, the FURPS+ model has been criticised for disregarding developer consideration  \cite{FURPS_drawbacks}, such as not taking into account portability and maintainability. 
Furthermore it has been pointed out, that our requirements fail to capture ...

TODO

\subsection{Use Cases}
TODO
\subsection{Feasibility Analysis}
TODO

\end{document}